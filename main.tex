\documentclass{article}
\usepackage[utf8]{inputenc}
\usepackage{datetime}

\title{A Review of Resource Management in Service Oriented Computing}
\author{Jovan Trujillo}
\date{\currenttime}

\usepackage{natbib}
\usepackage{graphicx}

\begin{document}

\maketitle

\section{Introduction}
Resource management and scheduling in a wide area service oriented computing network is a complex problem because resources are geographically distributed, heterogeneous, owned by different individuals and organizations with different policies, have various access and code models, and have dynamically varying loads and availability.\citep{economygrid} Systems addressing these problems have been developed and are called Grid Resource Brokers (GRB). One such system is called Nimrod/G.

% This is the introduction from economygrid...
Grid based computational infrastructure is a promising next generation computing platform for solving large-scale resource intensive problems. It couples a wide variety of geographically distributed computational resources (such as PCs, workstations, and clusters), storage systems, data sources, databases, computational kernels, and special purpose scientific instruments and presents them as a unified integrated resource. However, including application development, the management and scheduling of computations in the Grid environment is a complex undertaking as resources are geographically distributed, heterogeneous in nature, owned by different individuals or organizations with their own policies, have different access cost models with dynamically varying loads and availability conditions. A typical market-oriented Grid environment is shown in Figure 1. It encompasses a wide range of software technologies from local operating environments (operating or queuing systems) to global resource brokers and applications that are designed to exploit Grid capability. The interactions between these components must be secure and adapt to the changing resource status. Internationally, there are many projects actively exploring the design and development of different Grid system components and services for secure execution of applications on wide-area resources.

As shown in Figure 1, the users in global Grid environment essentially interact with a Grid Resource Broker (GRB) that hides the complexity of resource management and scheduling. The broker discovers resources using Grid Information Services (GIS), negotiates with grid-enabled resources or their agents for service costs, performs resource selection, maps and schedules tasks to resources, stages the application and data for processing on remote resources, and finally gathers results and hands them to the user. It is also responsible for monitoring application execution progress along with managing and adapting to changes in the Grid environment such as resource failures. 

In this paper we identify requirements of users (resource providers and consumers) in the Grid economy and various resource management issues that need to be addressed in realizing such a Grid system. We briefly discuss popular economic models for resource trading and present related work that employs computational economy in resource management. We propose a scalable architecture and new services for the Grid that provide mechanisms for addressing user requirements. The proposed architecture leverages services offered by the existing Grid systems such as Globus and offers new core services for resource trading. We discuss the use of these economic models and services for developing tools such as the Nimrod/G resource broker. We discuss a case study consisting of scheduling a prameter-sweep application on a large computational Grid spanning four continents and present some experimental results. 

% Grid Economy and Resource Management Issues
The current research and investment into computational grids is motivated by an assumption that coordinated access to diverse and geographically distributed resources is valuable. In this paradigm, we need mechanisms that allow such coordinated access, but also sustainable, scalable models and policies that promote precious Grid resource sharing. Based on the success of economic institutions in the real world as a sustainable model for exchanging and regulating resources, goods, and services, we propose a computational economy framework. Among other things, this framework provides a mechanism to indicate which users should receive priority. In [4], we have presented several arguments in favor of developing Grid architecture for computational economy and its benefits.

Like all systems involving goals, resources, and actions, computations can be viewed in economic terms. With the proliferation of networks, high-end computing systems architecture has moved from centralized toward decentralized models of control and action; use of economic driven market mechanisms would be a natural extension of this development. The ability of trade and price mechanisms to combine local decisions by diverse entities into globally effective characteristic implies their value for organizing computations in large systems such as Internet scale computational Grids. 

The two key players in market oriented computational Grid are resource providers (we refer hereafter as GSPs - Grid Service Providers) and resource consumers (we refer hereafter as GRBs - Grid Resource Broker that acts as a consumer's software agent). Both have their own expectations and strategies for being part of the Grid. In this Grid economy, resource consumers adopt the strategy of solving their problems at low cost within a required timeframe and resource providers adopt the strategy of obtaining best possible return on their investment. The resource owners try to maximize their resource utilization by offering a competitive service access cost in order to attract consumers. The users (resource consumers) have an option of choosing the providers that best meet their requirements. If resource providers have local users, they will try to recoup the best possible return on "idle/leftover" resources. In order to achieve this, the Grid systems need to offer tools and mechanisms that allow both resource providers and consumers to express their requirements. The Grid resource consumers interact with brokers (also called super-schedulers) to expres their requirements such as the budget that they are willing to invest for solving a given problem and a deadline, a timeframe by which they need results. They also need capability to trade between these two requirements and steer the computations accordingly. The Grid Service Providers need tools for expressing their pricing policies and mechanisms that help them to maximize the profit and resource utilization. Various economic models, ranging from commodity market to auction-based, can be adopted for deciding pricing strategies. The Grid infrastructure needs to support these economic models for resource trading.

To date, individuals or organizations that have contributed resources to the Grid have been largely motivated by the public good, prizes, fun, fame, or collaborative advantage. This is clearly evident from the construction of private grids (but on volunteer resources) or research test-beds such as Distributed.net, SETI@Home, Condor pool, DAS (Distributed ASCI Supercomputer), GUSTO, and eGrid. Even commercial companies such as Entropia, ProcessTree, Popular Power, Mojo Nation, United Devices, and Parabon are exploiting idle CPU cycles from desktop machines to build a commercial computational Grid. These companies are able to develop large-scale infrastructure for Internet computing and use it for their own financial gain by charging for access to CPU cycles for their customers without offering fiscal incentive to all resource contributors. In the long run, this model is less likely to succeed in creating a maintainable and sustainable infrastructure. Therefore, a Grid economy seems a better model for managing and handling requirements of both Grid providers and consumers. It is interesting to note that, even in electricity Grid, bid-based electricity trading over the Internet has been adopted to develop competitive forces in the electricity marketplace.

An economy approach to grid computing introduces a number of new issues to be addressed in addition to those already addressed by existing Grid systems. Grid toolkits such as Globus have addressed the five challenging resource management problems introduced by computational grids: site autonomy, heterogeneous substrate, policy extensibility, resource allocation or co-allocation, and online control. In [4], we proposed a "computational economy" as another key challenging issue that needs to be addressed for developing a service oriented Grid. We proposed an economy-based resource management architecture called GRACE (GRid Architecture for Computational Economy). The GRACE architecture is designed in such a way that it reuses or leverages services supported by the existing infrastructurees (such as Globus, Legion, Condor/G, QBank, and NetCash as much as possible. It offers new services that are particularly missing in them for contructing an economy Grid. The economy Grid framework needs to provide infrastructure that offers the following:
* An Information and Market directory for publicizing Grid entities
* Models for establishing the value of resources
* Resource pricing schemes and publishing mechanisms
* Economic models and negotiation protocols
* Mediators to act as a regulatory agency for establishing resource value, currency standards, and crisis handling. 
* Accounting, Billing, and Payment Mechanisms

% Economy Models and Related Work
A market-based approach in computational system design has been the topic of research over the years. Some of these systems have developed a substantial theoretical foundation but without large-scale deployment, experimental validation, and testing. A number of recent systems are attempting to apply computational economy for Web-based computing or for cluster-based systems. Research in the area of artificial intelligence and agents based computing has explored economy-based approach for migration of agents and resource allocation. FIPA (Foundation for Intelligent Physical Agents), a consortium of the software agents community, has proposed a specification for agents negotiation.

Various economic models for resource trading and establishing pricing strategies have been proposed and they include,
* A Commodity Market (Flat or Demand & Supply driven pricing) Model
* A Posted Price Model
* A Bargaining Model
* A Tendering/Contract-net Model
* An Auction Model
* A Bid-based Proportional Resource Sharing Model
* A Community/Coalition/Bartering Model

In [6], we presented architecture and issues associated in implementing the above economic models in the Grid environment. In the context of business negotiation on the Internet these models have been discussed in [23]. The resource providers and consumers can use any one or more of these economic models or even a combination of them while establishing access price depending upon their objective functions. Either the GSP or the GRB can initiate resource trading and participate in a market like environment depending on their requirements. In the commodity market model, resource providers competitively set the price and advertise their service in business directory as service providers (see Figure 1). The pricing scheme can be static or dynamic in nature. Consumers choose resource providers through cost-benefit analysis. The posted price model is similar to commodity market model except that it posts offers long before scheduling. In the bargaining model, providers and consumers do not influence the price for access to services. The negotiation happens privately between a consumer and a provider and there is no way for a consumer to know how much others value the resource services. Accordingly, the consumers need to decide whether to accept/reject offers depending on its private objective function. 

In the Tender/Contract Net model, the consumer (GRB) invites sealded bids from several GSPs and selects those bids that offer lowest service cost within their deadline and budget. In the Auction model, producers invite bids from many consumers and each bidder is free to raise their bid accordingly. The auction ends when no new bids are received. The auction can be performed through open or closed bidding protocols. In the bid-based proportional resource-sharing model, the amount of resource allocated to consumers is proportional to the value of their bids.

In the Community/Coalition/Bartering model, a group of individuals can create a cooperative computing environment to share each other's resources. Those who are contributing resources to a common pool can get access to resources when in need. A sophisticated model can also be employed for deciding the share of resources a contributor can obtain and can allow a user to accumulate credit for future needs. Systems like Mojonation.net employ this credit-based bartering model for storage sharing across the community network. This model works when all participants in the Grid are both service providers and consumers. 



% A Grid resource broker (GRB) is the main interface a client interacts with when using a distributed system. Grid Information Services (GIS) is used by the broker
% to discover services available for the client. 
% 

\bibliographystyle{plain}
\bibliography{references}
\end{document}
